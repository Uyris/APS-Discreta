\documentclass[12pt,a4paper]{article}
\usepackage[utf8]{inputenc}
\usepackage[brazil]{babel}
\usepackage{amsmath, amssymb}
\usepackage{geometry}
\geometry{margin=2.5cm}
\usepackage{enumitem}
\usepackage{hyperref}
\usepackage{array}
\usepackage{booktabs}
\usepackage{xcolor}
\usepackage{framed}
\usepackage{graphicx}

\begin{document}

\begin{center}
\Large\textbf{APS -- Lógica e Matemática Discreta 2025.2} \\
\vspace{0.3cm}
\large Professor: Fillipe Resina \\
\vspace{0.5cm}
\textbf{Modelando um Mundo com Lógica de Primeira Ordem: Warcraft III e Expansões} \\
\vspace{0.5cm}
\textbf{Integrantes:} Matheus Luciano Alves de Oliveira Silva \\ Yuri Henrique da Cunha Santos
\end{center}

\section*{1. Objetivo}
Aplicar os conceitos de Lógica de Primeira Ordem para representar formalmente um cenário fictício, baseado na história de {Warcraft III: Reign of Chaos} e {The Frozen Throne}, utilizando predicados, quantificadores e regras de dedução natural. Em seguida, o modelo foi implementado em Prolog para teste de inferências automáticas.

\section*{2. Cenário: O Mundo de Warcraft III}
O universo de Warcraft apresenta diversos agentes e entidades com relações complexas de poder, corrupção e alianças. Nesta modelagem, consideramos personagens principais: {Arthas, Illidan, Thrall, Jaina, Sylvanas, Kel'Thuzad, Mal'Ganis, Archimonde, Lady Vashj} e objetos relevantes: { Frostmourne, Frozen Throne, Tome of Power}.\newline
A narrativa centra-se na corrupção de Arthas, a busca de poder de Illidan e o conflito entre raças (Humanos, Orcs, Elfos Noturnos, Mortos-Vivos e Demônios).

\section*{3. Predicados, Funções e Constantes}
\subsection*{Predicados}
\begin{itemize}[leftmargin=1.5cm]
  \item human(x): x é humano.
  \item orc(x): x é orc.
  \item nightelf(x): x é elfo noturno.
  \item undead(x): x é morto-vivo.
  \item demon(x): x é demônio.
  \item hero(x): x é herói.
  \item king(x): x é rei.
  \item wields(x,o): x empunha o objeto o.
  \item corrupted\_by(x,y): x foi corrompido por y.
  \item killed(x,y): x matou y.
  \item seeks(x,o): x busca o objeto o.
  \item commands(x,y): x comanda y.
  \item ally(x,y): x é aliado de y.
  \item leader\_of(x,f): x é líder da facção f.
  \item became(x,t): x tornou-se t (mudança de papel).
  \item artifact(o): o é artefato.
\end{itemize}

\subsection*{Constantes}
\textit{arthas, illidan, thrall, jaina, sylvanas, kelthuzad, malganis, archimonde, frostmourne, frozenthrone, tome\_of\_power, king\_terenas, scourge, nightelves, lich\_king, lich\_figure.}

\section*{4. Fórmulas em Lógica de Primeira Ordem}
\begin{enumerate}
  \item human(arthas)
  \item hero(arthas)
  \item artifact(frostmourne)
  \item $\forall x (wields(x, frostmourne) \rightarrow corrupted\_by(x, frostmourne))$
  \item wields(arthas, frostmourne)
  \item $\forall x (corrupted\_by(x, frostmourne) \rightarrow became(x, lich\_figure))$
  \item $\forall x\forall y (killed(x,y) \wedge king(y) \rightarrow traitor(x))$
  \item king(king\_terenas), killed(arthas, king\_terenas)
  \item $\forall x (demon(x) \rightarrow \neg hero(x))$
  \item $\exists x (became(x, lich\_king) \wedge commands(x, undead\_faction))$
  \item seeks(illidan, tome\_of\_power)
  \item $\forall x (seeks(x, tome\_of\_power) \wedge artifact(tome\_of\_power) \rightarrow \neg ally(x, nightelves))$
  \item undead(kelthuzad)
  \item $leader\_of(arthas, scourge) \rightarrow commands(arthas, scourge)$
\end{enumerate}

\section*{5. Dedução Natural}
\subsection*{Dedução A — Arthas torna-se uma figura Lich}
\textbf{Premissas:}
\begin{enumerate}[label=(\arabic*)]
  \item $\forall x (wields(x, frostmourne) \rightarrow corrupted\_by(x, frostmourne))$
  \item wields(arthas, frostmourne)
  \item $\forall x (corrupted\_by(x, frostmourne) \rightarrow became(x, lich\_figure))$
\end{enumerate}
\textbf{Prova:}
\begin{enumerate}
  \item Pela (1), instanciando para Arthas: $wields(arthas, frostmourne) \rightarrow corrupted\_by(arthas, frostmourne)$
  \item De (2) e (1) por Modus Ponens: $corrupted\_by(arthas, frostmourne)$
  \item Pela (3), instanciando para Arthas: $corrupted\_by(arthas, frostmourne) \rightarrow became(arthas, lich\_figure)$
  \item De (2) e (3) por Modus Ponens: $became(arthas, lich\_figure)$
\end{enumerate}
\textbf{Conclusão:} Arthas tornou-se uma figura Lich.

\vspace{0.5cm}
\noindent\textbf{Figura 1: Representação gráfica da Dedução A}

\begin{center}
\small
\begin{tabular}{c>{\raggedright\arraybackslash}p{8cm}>{\raggedright\arraybackslash}p{3cm}}
\toprule
\textbf{Passo} & \textbf{Fórmula} & \textbf{Justificativa} \\
\midrule
1. & $\forall x (wields(x, frostmourne) \rightarrow corrupted\_by(x, frostmourne))$ & Premissa (P1) \\[0.3cm]
2. & $wields(arthas, frostmourne)$ & Premissa (P2) \\[0.3cm]
3. & $\forall x (corrupted\_by(x, frostmourne) \rightarrow became(x, lich\_figure))$ & Premissa (P3) \\[0.3cm]
4. & $wields(arthas, frostmourne) \rightarrow corrupted\_by(arthas, frostmourne)$ & $\forall$E 1; \newline $x:=arthas$ \\[0.3cm]
5. & $corrupted\_by(arthas, frostmourne)$ & MP 2,4 \\[0.3cm]
6. & $corrupted\_by(arthas, frostmourne) \rightarrow became(arthas, lich\_figure)$ & $\forall$E 3; \newline $x:=arthas$ \\[0.3cm]
7. & $became(arthas, lich\_figure)$ & MP 5,6 \\[0.3cm]
\bottomrule
\end{tabular}
\end{center}

\subsection*{Dedução B — Arthas é um traidor}
\textbf{Premissas:}
\begin{enumerate}[label=(\arabic*)]
  \item $\forall x\forall y (killed(x,y) \wedge king(y) \rightarrow traitor(x))$
  \item killed(arthas, king\_terenas)
  \item king(king\_terenas)
\end{enumerate}
\textbf{Prova:}
\begin{enumerate}
  \item Pela (2) e (3): $killed(arthas, king\_terenas) \wedge king(king\_terenas)$
  \item Pela (1), instanciando Arthas e Terenas: $killed(arthas, king\_terenas) \wedge king(king\_terenas) \rightarrow traitor(arthas)$
  \item De (1) e (2) por Modus Ponens: $traitor(arthas)$
\end{enumerate}
\textbf{Conclusão:} Arthas é um traidor.

\vspace{0.5cm}
\noindent\textbf{Figura 2: Representação gráfica da Dedução B}

\begin{center}
\small
\begin{tabular}{c>{\raggedright\arraybackslash}p{8cm}>{\raggedright\arraybackslash}p{3cm}}
\toprule
\textbf{Passo} & \textbf{Fórmula} & \textbf{Justificativa} \\
\midrule
1. & $\forall x\forall y (killed(x,y) \wedge king(y) \rightarrow traitor(x))$ & Premissa (P1) \\[0.3cm]
2. & $killed(arthas, king\_terenas)$ & Premissa (P2) \\[0.3cm]
3. & $king(king\_terenas)$ & Premissa (P3) \\[0.3cm]
4. & $killed(arthas, king\_terenas) \wedge king(king\_terenas)$ & $\wedge$I 2,3 \\[0.3cm]
5. & $(killed(arthas, king\_terenas) \wedge king(king\_terenas)) \rightarrow$ \newline $traitor(arthas)$ & $\forall$E 1; \newline $x:=arthas$, \newline $y:=king\_terenas$ \\[0.3cm]
6. & $traitor(arthas)$ & MP 4,5 \\[0.3cm]
\bottomrule
\end{tabular}
\end{center}

\vspace{0.5cm}
\noindent\textbf{Figura 3: Captura de tela das consultas em Prolog - Parte 1}

\begin{center}
\includegraphics[width=0.85\textwidth,keepaspectratio]{consulta1.jpeg}
\end{center}

\vspace{0.5cm}
\noindent\textbf{Figura 4: Captura de tela das consultas em Prolog - Parte 2}

\begin{center}
\includegraphics[width=0.85\textwidth,keepaspectratio]{consulta2.jpeg}
\end{center}

\vspace{0.5cm}
\noindent\textbf{Figura 5: Captura de tela das consultas em Prolog - Parte 3}

\begin{center}
\includegraphics[width=0.85\textwidth,keepaspectratio]{consulta3.jpeg}
\end{center}

\end{document}
